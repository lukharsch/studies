\documentclass[a4paper,12pt]{article}
\usepackage{fancyhdr}
\usepackage[ngerman,german]{babel}
\usepackage{german}
\usepackage[utf8]{inputenc}
\usepackage[active]{srcltx}
%\usepackage{algorithm}
%\usepackage[noend]{algorithmic}
\usepackage{amsmath}
\usepackage{amssymb}
\usepackage{amsthm}
\usepackage{bbm}
\usepackage{enumerate}
\usepackage{graphicx}
\usepackage{ifthen}
\usepackage{listings}
\usepackage{struktex}
\usepackage{hyperref}
\usepackage{tabularx}
\usepackage[linesnumbered, ruled]{algorithm2e}
\SetKwRepeat{Do}{do}{while}

\newcommand{\Fach}{Berechenbarkeit und Komplexität}
\newcommand{\Name}{Lukas Harsch}
\newcommand{\Matrikelnummer}{979272}
\newcommand{\Semester}{Sommersemester 2019}
\newcommand{\Kapitel}{5}
\newcommand{\Titel}{}

\setlength{\parindent}{0em}
\topmargin 0cm
\oddsidemargin 0cm
\evensidemargin 0cm
\setlength{\textheight}{9.2in}
\setlength{\textwidth}{6.0in}

\hypersetup{
	pdftitle={\Fach{}: "Ubungsblatt \Kapitel{}},
	pdfauthor={\Name},
	pdfborder={0 0 0}
}

%\lstset{ %
%	language=java,
%	basicstyle=\footnotesize\tt,
%	showtabs=false,
%	tabsize=2,
%	captionpos=b,
%	breaklines=true,
%	extendedchars=true,
%	showstringspaces=false,
%	flexiblecolumns=true,
%}

\title{"Ubungsblatt \Kapitel}
\author{\Name{}}

\begin{document}
	\thispagestyle{fancy}
	\lhead{\sf \small \Name{}}
	\chead{\sf \small \Fach}
	\rhead{\sf \small \Semester{}}
	\rfoot{\sf \tiny Keine Gewähr auf Richtigkeit und Vollständigkeit}
	\lfoot{\sf \tiny CC BY-NC-SA}
	\begin{center}
		\LARGE \sf \textbf{"Ubungsblatt \Kapitel}\\
		\LARGE \sf \Titel
	\end{center}
	\vspace*{0.2cm}
	
	\section*{Aufgabe 2}
	\subsubsection*{a)}
	$\overline{A}$ ist entscheidbar
	\begin{align*}
		A & \Rightarrow x \mapsto \begin{cases}
			1, x \in A \\0, x \notin A
		\end{cases}\\
		\overline{A} & \Rightarrow x \mapsto \begin{cases}
		1, x \notin A \\0, x \in A
		\end{cases}
	\end{align*}
	
	\subsubsection*{b)}
	$A \cup B$ ist entscheidbar
	\begin{center}
		$\mathcal{X}_{A \cup B} \left(w\right) = \mathcal{X}_{A} \left(w\right) \cup \mathcal{X}_{B} \left(w\right)$\\
		
		$\Rightarrow$ Charakteristische Funktion ist berechenbar, also entscheidbar
	\end{center}

	\subsubsection*{c)}
	$AB$ ist entscheidbar
	\begin{center}
		$\mathcal{X}_{AB} \left(w\right) = \underbrace{\max \left\lbrace \underbrace{\mathcal{X}_{A} \left(y\right) \wedge \mathcal{X}_{B} \left(z\right)}_{= 1\text{ oder }0} | \underbrace{x = yz \text{und} y,z \in \Sigma ^{*}}_{:= b1}\right\rbrace}_{\text{Nur 1, wenn }y\text{ in }A\text{ \textbf{und} }z\text{ in }B\text{ liegt mit Bedingung }b1}$\\
		
		$\Rightarrow$ Charakteristische Funktion ist berechenbar, also entscheidbar
	\end{center}
	
	\subsubsection*{d)}
	$A^*$ ist entscheidbar
	\begin{align*}
		&A^* := \bigcup \limits_{n \geq 0} A^n\\
		&A^n := \left\lbrace w | w = w_1,w_2,\cdots,w_n \text{ mit } w_i \in A\right\rbrace
	\end{align*}
	\begin{center}
		$\mathcal{X}_{A^*} \left(x\right) = \bigvee \limits_{n=1} ^{\left|x\right|} \bigvee \limits_{x=x_1,\cdots,x_n} \bigwedge \limits_{i=1}^{n} \mathcal{X}_{A} \left(x_i\right)$
	\end{center}
	\newpage
	
	\subsubsection*{e)}
	$C \cup D$ ist semi-entscheidbar\\
	
	Sei $M_C$ der Entscheidungsalgorithmus für die Sprache $C$, analog für die Sprache $D$
	\begin{algorithm}[H]
		\KwData{Input $x$}
		\For{$s=1,2,\cdots$}{
			\If{$M_C$ hält bei Eingabe $x$ in $s$ Schritten}{
				Output 1\;
				Stop\;
			}
			\If{$M_D$ hält bei Eingabe $x$ in $s$ Schritten}{
				Output 1\;
				Stop\;
			}
		}
		\caption{$C \cup D$ ist semi-entscheidbar}
	\end{algorithm}
	$\Rightarrow$ Hier muss nur entweder $M_C$ oder $M_D$ 1 sein, wegen der Semi-Entscheidbarkeit
	
	\subsubsection*{f)}
	$C \cap D$ ist semi-entscheidbar\\
	
	Sei $M_C$ der Entscheidungsalgorithmus für die Sprache $C$, analog für die Sprache $D$
	\begin{algorithm}[H]
		\KwData{Input $x$}
		\For{$s=1,2,\cdots$}{
			\If{$M_C$ hält bei Eingabe $x$ in $s$ Schritten}{
				\If{$M_D$ hält bei Eingabe $x$ in $s$ Schritten}{
					Output 1\;
					Stop\;
				}
			}
		}
		\caption{$C \cap D$ ist semi-entscheidbar}
	\end{algorithm}
	$\Rightarrow$ Hier muss der Algorithmus für $M_C$ und $M_D$ in abzählbar vielen Schritte mit der Eingabe $x$ halten, sodass der Algorithmus eine 1 ausgibt
	\newpage
	
	\section*{Aufgabe 3}
	Zu zeigen: $f\left(n\right)$ berechenbar und $L$ entscheidbar
\end{document}