\documentclass[a4paper,12pt]{article}
\usepackage{fancyhdr}
\usepackage[ngerman,german]{babel}
\usepackage{german}
\usepackage[utf8]{inputenc}
\usepackage[active]{srcltx}
%\usepackage{algorithm}
%\usepackage[noend]{algorithmic}
\usepackage{amsmath}
\usepackage{amssymb}
\usepackage{amsthm}
\usepackage{bbm}
\usepackage{enumerate}
\usepackage{graphicx}
\usepackage{ifthen}
\usepackage{listings}
\usepackage{struktex}
\usepackage{hyperref}
\usepackage{tabularx}
\usepackage[linesnumbered, ruled]{algorithm2e}
\SetKwRepeat{Do}{do}{while}

\newcommand{\Fach}{Berechenbarkeit und Komplexität}
\newcommand{\Name}{Lukas Harsch}
\newcommand{\Matrikelnummer}{979272}
\newcommand{\Semester}{Sommersemester 2019}
\newcommand{\Kapitel}{6}
\newcommand{\Titel}{}

\setlength{\parindent}{0em}
\topmargin 0cm
\oddsidemargin 0cm
\evensidemargin 0cm
\setlength{\textheight}{9.2in}
\setlength{\textwidth}{6.0in}

\hypersetup{
	pdftitle={\Fach{}: "Ubungsblatt \Kapitel{}},
	pdfauthor={\Name},
	pdfborder={0 0 0}
}

%\lstset{ %
%	language=java,
%	basicstyle=\footnotesize\tt,
%	showtabs=false,
%	tabsize=2,
%	captionpos=b,
%	breaklines=true,
%	extendedchars=true,
%	showstringspaces=false,
%	flexiblecolumns=true,
%}

\title{"Ubungsblatt \Kapitel}
\author{\Name{}}

\begin{document}
	\thispagestyle{fancy}
	\lhead{\sf \small \Name{}}
	\chead{\sf \small \Fach}
	\rhead{\sf \small \Semester{}}
	\rfoot{\sf \tiny Keine Gewähr auf Richtigkeit und Vollständigkeit}
	\lfoot{\sf \tiny CC BY-NC-SA}
	\begin{center}
		\LARGE \sf \textbf{"Ubungsblatt \Kapitel}\\
		\LARGE \sf \Titel
	\end{center}
	\vspace*{0.2cm}
	
	\section*{Aufgabe 1}
	$A$ rekursive Aufzählung $\Rightarrow$ $A$ semi-entscheidbar
	
	\section*{Aufgabe 2}
	\subsubsection*{a)}
	$K \preccurlyeq K \times \overline{K}$\\
	
	
	$K$ ist nicht entscheidbar.\\
	Es muss gezeigt werden:
	\begin{align*}
		&f: \left\lbrace 0,1\right\rbrace^* \rightarrow \left\lbrace 0,1\right\rbrace^* * \left\lbrace 0,1\right\rbrace^*\\
		&w \in K \Leftrightarrow f\left( w \right) \in K \times \overline{K}\\
		&k \subset \left\lbrace 0,1\right\rbrace^*\\
		&y \in \overline{K} \neq \emptyset\\
		&f\left( w\right) = \left(w,y\right) \forall w \in \left\lbrace 0,1\right\rbrace^*
	\end{align*}
	\subsubsection*{b)}
	$\overline{K}$ ist nicht reduzierbar auf $K$\\
	
	$K$ ist semi-entscheidbar
	\begin{algorithm}
		\KwData{Eingabe $w$}
		\For{$s = 1,2,3, \cdots$}{for-block}
		
		\caption{$\overline{K}$ ist nicht reduzierbar auf $K$}
	\end{algorithm}
	
	\subsubsection*{c)}
	
	\section*{Aufgabe 3}
	\subsubsection*{a)}
	\underline{Ja}, weil ein Programm angegeben werden kann bzw. $f$ in endlichen Schritten hält.
	\subsubsection*{b)}
	\underline{Ja}, denn $x \neq y \Rightarrow f\left(x\right) \neq f\left(y\right)$
	\subsubsection*{c)}
	$M_{f\left( x \right)} \left( y\right) = \begin{cases}
		1 \text{, falls } x \neq y\\
		0 \text{, sonst}
	\end{cases}$
	\subsubsection*{d)}
	$K \cap \left\lbrace f \left( i  \right) : i \in \mathbb{N} \right\rbrace = \left\lbrace f\left( i\right) : i \in \mathbb{N} \right\rbrace$\\
	\hspace*{1cm} weil $f\left( i\right)$ genau dann mit 1 hält, wenn es auf sich selbst hält

\end{document}