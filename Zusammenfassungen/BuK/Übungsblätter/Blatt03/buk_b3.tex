\documentclass[a4paper,12pt]{article}
\usepackage{fancyhdr}
\usepackage[ngerman,german]{babel}
\usepackage{german}
\usepackage[utf8]{inputenc}
\usepackage[active]{srcltx}
%\usepackage{algorithm}
%\usepackage[noend]{algorithmic}
\usepackage{amsmath}
\usepackage{amssymb}
\usepackage{amsthm}
\usepackage{bbm}
\usepackage{enumerate}
\usepackage{graphicx}
\usepackage{ifthen}
\usepackage{listings}
\usepackage{struktex}
\usepackage{hyperref}
\usepackage{tabularx}
\usepackage[linesnumbered, ruled]{algorithm2e}
\SetKwRepeat{Do}{do}{while}

\newcommand{\Fach}{Berechenbarkeit und Komplexität}
\newcommand{\Name}{Lukas Harsch}
\newcommand{\Matrikelnummer}{979272}
\newcommand{\Semester}{Sommersemester 2019}
\newcommand{\Kapitel}{3}
\newcommand{\Titel}{}

\setlength{\parindent}{0em}
\topmargin 0cm
\oddsidemargin 0cm
\evensidemargin 0cm
\setlength{\textheight}{9.2in}
\setlength{\textwidth}{6.0in}

\hypersetup{
	pdftitle={\Fach{}: "Ubungsblatt \Kapitel{}},
	pdfauthor={\Name},
	pdfborder={0 0 0}
}

%\lstset{ %
%	language=java,
%	basicstyle=\footnotesize\tt,
%	showtabs=false,
%	tabsize=2,
%	captionpos=b,
%	breaklines=true,
%	extendedchars=true,
%	showstringspaces=false,
%	flexiblecolumns=true,
%}

\title{"Ubungsblatt \Kapitel}
\author{\Name{}}

\begin{document}
	\thispagestyle{fancy}
	\lhead{\sf \small \Name{}}
	\chead{\sf \small \Fach}
	\rhead{\sf \small \Semester{}}
	\rfoot{\sf \tiny Keine Gewähr auf Richtigkeit und Vollständigkeit}
	\lfoot{\sf \tiny CC BY-NC-SA}
	\begin{center}
		\LARGE \sf \textbf{"Ubungsblatt \Kapitel}\\
		\LARGE \sf \Titel
	\end{center}
	\vspace*{0.2cm}
	
	\section*{Aufgabe 1}
	\subsubsection*{a)}
	\begin{algorithm}[H]
		\KwData{$e = 1$, $x = 1$\;}
		\ForEach{n}{e = e $\cdot$ x\; x = x + 1\;}
		\caption{Fakultät}
	\end{algorithm}
	\subsubsection*{b)}
	\begin{algorithm}[H]
		\KwData{$k = 1$\;}
		\ForEach{$x_1$}{
			k = 0\;
		}
		\ForEach{k}{
			A\;
		}
		k = 0\;
		\ForEach{$x_1$}{
			k = 1\;
		}
		\ForEach{k}{
			B\;
		}
		\caption{IF-Anweisung}
	\end{algorithm}
	\subsubsection*{c)}
	\begin{algorithm}[H]
		\KwData{$y = m-n$, $x_1 = 1$\;}
		\ForEach{y}{
			$x_1$ = 0\;
		}
		\ForEach{$x_1$}{
			y = n-m\;
		}
		\caption{Minus-Operator}
	\end{algorithm}
	\subsubsection*{d)}
	\begin{algorithm}[H]
		\KwData{$x_1 = x_2$, $x_4 = x_3 - x_1$}
		\ForEach{$x_4$}{
			P\;
		}
		\caption{FOR-Schleife}
	\end{algorithm}
	\subsubsection*{e)}
	\begin{algorithm}[H]
		\KwData{$n$}
		\For{$x_1$ = 2 \KwTo{n-1}}{
			\If{n MOD $x_1$}{0\;}
		}
		1\;
		\caption{Primzahltest}
	\end{algorithm}
	
	\section*{Aufgabe 2}
	Sei $g\left(k, n\right)$ LOOP-berechenbar, dann existiert ein $P$ mit $x \in \mathbb{N}$ LOOP-Schleifen, welches $\underbrace{g\left(k, n\right)}_{k \in \mathbb{N^+}, n \in \mathbb{N}}$ berechnet.\\
	Es gilt $g\left(x+1, n\right) = f_{x+1}\left(n\right)$.\\
	$f_{x+1}\left(n\right)$ kann durch $P$ mit $x+1$ LOOP-Schleifen berechnet werden, aber nicht durch $P$ mit $x$ LOOP-Schleifen berechnet werden.\\
	$\Rightarrow$ Widerspruch zur Annahme
	
	\section*{Aufgabe 3}
	\begin{algorithm}[H]
		\KwData{counter := 0}
		\While{n $\neq$ 0}{
			\If{n MOD 2 = 1}{counter := counter + 1 \;}
			
			
			n := n DIV 2\;
		}
		
		\eIf{counter MOD 2 = 0}{
			$x_0$ := 1\;
		}{
			$x_0$ = 0\;
		}
		\caption{WHILE-Programm}
	\end{algorithm}
	Das Ergebnis steht in $x_0$.
	
\end{document}