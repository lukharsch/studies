\documentclass[a4paper,12pt]{article}
\usepackage{fancyhdr}
\usepackage[ngerman,german]{babel}
\usepackage{german}
\usepackage[utf8]{inputenc}
\usepackage[active]{srcltx}
\usepackage{algorithm}
\usepackage[noend]{algorithmic}
\usepackage{amsmath}
\usepackage{amssymb}
\usepackage{amsthm}
\usepackage{bbm}
\usepackage{enumerate}
\usepackage{graphicx}
\usepackage{ifthen}
\usepackage{listings}
\usepackage{struktex}
\usepackage{hyperref}
\usepackage{tabularx}

\newcommand{\Fach}{Berechenbarkeit und Komplexität}
\newcommand{\Name}{Lukas Harsch}
\newcommand{\Matrikelnummer}{979272}
\newcommand{\Semester}{Sommersemester 2019}
\newcommand{\Kapitel}{1}
\newcommand{\Titel}{}

\setlength{\parindent}{0em}
\topmargin 0cm
\oddsidemargin 0cm
\evensidemargin 0cm
\setlength{\textheight}{9.2in}
\setlength{\textwidth}{6.0in}

\hypersetup{
	pdftitle={\Fach{}: "Ubungsblatt \Kapitel{}},
	pdfauthor={\Name},
	pdfborder={0 0 0}
}

\lstset{ %
	language=java,
	basicstyle=\footnotesize\tt,
	showtabs=false,
	tabsize=2,
	captionpos=b,
	breaklines=true,
	extendedchars=true,
	showstringspaces=false,
	flexiblecolumns=true,
}

\title{"Ubungsblatt \Kapitel}
\author{\Name{}}

\begin{document}
	\thispagestyle{fancy}
	\lhead{\sf \small \Name{}}
	\chead{\sf \small \Fach}
	\rhead{\sf \small \Semester{}}
	\rfoot{\sf \tiny Keine Gewähr auf Richtigkeit und Vollständigkeit}
	\lfoot{\sf \tiny CC BY-NC-SA}
	\begin{center}
		\LARGE \sf \textbf{"Ubungsblatt \Kapitel}\\
		\LARGE \sf \Titel
	\end{center}
	\vspace*{0.2cm}
	
	\section*{Aufgabe 1}
	$M_n$, $\forall _n \in \mathbb{N}$ abzählbar hei"st, dass es $\forall n \in \mathbb{N}$ nach Voraussetzung eine surjektive Abbildung gibt mit $f_n$: $\mathbb{N} \Rightarrow M_n$.\\
	
	Wir ordnen die Funktionswerte wie folgt an: 
	\begin{center}
		$\begin{array}{cccc}		
			f_1\left( 1\right) & f_1\left( 2\right) & f_1\left( 3\right) & f_1\left( 4\right)\\
			f_2\left( 1\right) & f_2\left( 2\right) & f_2\left( 3\right) & \\
			f_3\left( 1\right) & f_3\left( 2\right) & &\\
		\end{array}$
	\end{center}
	Mit
	\begin{center}
		$f_1\left( 1\right) \rightarrow f_1\left( 2\right) \rightarrow f_2\left( 1\right) \rightarrow f_3\left( 1\right) \rightarrow f_2\left( 2\right) \rightarrow f_1\left( 3\right) \rightarrow f_1\left( 4\right) \rightarrow f_2\left( 3\right) \rightarrow f_3\left( 2\right)$ usw...
	\end{center}
	Wir definieren eine Abbildung
	\begin{center}
		$f: \mathbb{N} \rightarrow \bigcup\limits_{n \in \mathbb{N}} M_n $\\
		
		demnach $f\left( 1\right) = f_1\left( 1\right)$, $f\left( 2\right) = f_1\left( 2\right)$, $f\left( 3\right) = f_2\left( 1\right)$, $f\left( 4\right) = f_3\left( 1\right)$, usw...
	\end{center}
	\textit{\underline{Behauptung}}: $f$ ist surjektiv\\
	\textit{\underline{Beweis}}: Sei $x \in \bigcup\limits_{n \in \mathbb{N}} M_n$, d.h. $\exists n \in \mathbb{N}$, $x \in M$\\
	$\Rightarrow$ Es gibt ein $l \in \mathbb{N}$ mit $f_n\left( l\right) = x$. Nach dieser Konstruktion gibt es $m \in \mathbb{N}: f\left( m\right) = f_n\left( l\right) = x$. Somit ist die Vereinigung abzählbarer Mengen wieder abzählbar
	
	\section*{Aufgabe 2}
	\subsubsection*{a)}
	Zu zeigen: $\sum_{k}^{*} = \left\lbrace 1, ..., k\right\rbrace^*$ abzählbar.\\
	
	Es sei $K$ aus $\mathbb{N}^+$ beliebig
	\begin{center}
		$\sum_{K}^{*} = \bigcup\limits_{n \in \mathbb{N}} \sum_{K}^{n}$
	\end{center}
	Da es nur endliche Strings der Länge n gibt (also $|\sum_{k}|^n$ viele), ist $\sum_{K}^{n}$ abzählbar.
	\begin{center}
		$\Rightarrow \sum_{K}^{*}$ abzählbar aus Aufgabe 1
	\end{center}

	\subsubsection*{b)}
	Es gilt: Ein Polynom $p\left( x\right)$ des $n$-ten Grades hat höchstens $n$-viele Nullstellen. Die Anzahl der Nullstellen eines Polynoms ist abzählbar. Da $a_i \in \mathbb{Z}$ mit $i \in \left\lbrace a, ..., n\right\rbrace$ aus den ganzen Zahlen kommt, ist die Menge aller Polynome mit ganzzahligen Koeffizienten abzählbar. Daraus und aus Aufgabe 1 folgt, dass die Vereinigung der Polynome abzählbar ist.
	\begin{center}
		$\Rightarrow$ Die Menge der reelen algebraischen Zahlen ist abzählbar
	\end{center}

	\subsubsection*{c)}
	Die Menge der reelen transzenten Zahlen $\mathbb{T}$ ist mit $\mathbb{T} = \mathbb{R} \backslash \mathbb{A}$ definiert, wobei $\mathbb{A}$ die Menge aller reelen algebraischen Zahlen ist.\\
	Da $\mathbb{R}$ überabzählbar ist und eine abzählbare Menge entfernt wird, muss $\mathbb{T}$ überabzählbar sein.

	\section*{Aufgabe 3}
	\subsubsection*{a)}
	\begin{align*}
		\delta\left( z_0, \square\right) &= \left( z_1, 1, R\right)\\
		\delta\left( z_0, 1\right) &= \left( z_1, 1, L\right)\\
		\delta\left( z_1, \square\right) &= \left( z_0, 1, L\right)\\
		\delta\left( z_1, 1\right) &= \left( z_e, 1, L\right)
	\end{align*}
	\begin{center}
		$z_0 \square \rightarrow 1 z_1 \square \rightarrow z_0 1 1 \square \rightarrow z_1 \square 1 1 \square \rightarrow z_0 \square 1 1 1 \rightarrow 1 z_1 1 1 1 \rightarrow z_e 1 1 1 1$\\
		
		$\Rightarrow BBTM\left( 2\right) = 4$
	\end{center}

	\subsubsection*{b)}
	\begin{center}
		$\left( 2 \cdot 2 \cdot \left( n + 1\right)\right)^{2n}$
	\end{center}
	
	\subsubsection*{c)}
	\begin{align*}
		\left( z_n, \square \right) &\rightarrow \left(z_e, 1, R\right)\\
		\left( z_n, 1 \right) &\rightarrow \left(z_n, 1, R\right)
	\end{align*}
	\begin{center}
		$\Rightarrow$ Somit kommt in jedem neuen Schritt mindestens eine $1$ (also 1 Holzstück) hinzu.\\
		$\Rightarrow BB\left( \cdot\right)$ ist streng monoton wachsende Funktion
	\end{center}
\end{document}