\documentclass[a4paper,12pt]{article}
\usepackage{fancyhdr}
\usepackage[ngerman,german]{babel}
\usepackage{german}
\usepackage[utf8]{inputenc}
\usepackage[active]{srcltx}
\usepackage{algorithm}
\usepackage[noend]{algorithmic}
\usepackage{amsmath}
\usepackage{amssymb}
\usepackage{amsthm}
\usepackage{bbm}
\usepackage{enumerate}
\usepackage{graphicx}
\usepackage{ifthen}
\usepackage{listings}
\usepackage{struktex}
\usepackage{hyperref}
\usepackage{tabularx}

\newcommand{\Fach}{Berechenbarkeit und Komplexität}
\newcommand{\Name}{Lukas Harsch}
\newcommand{\Matrikelnummer}{979272}
\newcommand{\Semester}{Sommersemester 2019}
\newcommand{\Kapitel}{2}
\newcommand{\Titel}{}

\setlength{\parindent}{0em}
\topmargin 0cm
\oddsidemargin 0cm
\evensidemargin 0cm
\setlength{\textheight}{9.2in}
\setlength{\textwidth}{6.0in}

\hypersetup{
	pdftitle={\Fach{}: Blatt \Kapitel{}},
	pdfauthor={\Name},
	pdfborder={0 0 0}
}

\lstset{ %
	language=java,
	basicstyle=\footnotesize\tt,
	showtabs=false,
	tabsize=2,
	captionpos=b,
	breaklines=true,
	extendedchars=true,
	showstringspaces=false,
	flexiblecolumns=true,
}

\title{Kapitel \Kapitel}
\author{\Name{}}

\begin{document}
	\thispagestyle{fancy}
	\lhead{\sf \small \Name{}}
	\chead{\sf \small \Fach}
	\rhead{\sf \small \Semester{}}
	\rfoot{\sf \tiny Keine Gewähr auf Richtigkeit und Vollständigkeit}
	\lfoot{\sf \tiny CC BY-NC-SA}
	\begin{center}
		\LARGE \sf \textbf{"Ubungsblatt \Kapitel}\\
		\LARGE \sf \Titel
	\end{center}
	\vspace*{0.2cm}
	
	\section*{Aufgabe 1}
	Sei $M := \lbrace x \in \left[ 0,1 \right] : x$ hat eine Dezimaldarstellung der Form $x = 0,a_0,a_1,a_2,...$ mit $a_i \in \lbrace 0,2 \rbrace$ für alle $i \in \mathbb{N} \rbrace$ Zeigen Sie, dass $M$ überabzählbar ist.\\
	
	Schreibe alle $x_n$ untereinander:
	\begin{center}
		$x_0 = 0,a_{00},a_{01},a_{02},a_{03}...$\\
		$x_1 = 0,a_{10},a_{11},a_{12},a_{13}...$\\
		$x_2 = 0,a_{20},a_{21},a_{22},a_{23}...$\\
		$x_3 = 0,a_{30},a_{31},a_{32},a_{33}...$\\
		$\vdots$
	\end{center}
	Jetzt lässt sich eine Zahl $b$ mit $b \in M$ und $b = 0,b_0,b_1,b_2,b_3,...$ bilden , wobei alle Dezimalstellen $b_i$ so gewählt werden, dass $b_i \neq a_{ii}$. Wenn beispielsweise $a_{11}$ also $2$ ist, so muss $b_1 = 0$.\\
	Nach diesem Prinzip kann immer eine Zahl generiert werden, die noch nicht in $M$ enthalten ist. Damit ist $M$ überabzählbar.
	
	\section*{Aufgabe 2}
	\subsubsection*{a)}
	Startwert: $011010$\\

	\begin{align*}
		z_0011010 &\rightarrow &\square z_011010 &\rightarrow &\square \square z_01010 &\rightarrow & \square \square \square z_0010\\
		z_0 \square &  &0 z_0 \square & &0 z_0 \square & &0 z_0 \square\\
		\square \square \square \square z_010 &\rightarrow &\square \square \square \square \square z_0 0 &\rightarrow &\square \square \square \square \square \square z_0 \square &\rightarrow & \square \square \square \square \square \square z_1 \square\\
		00 z_0 \square &  &00 z_0 \square & &000 z_0 \square & &00 z_10 \square\\
		\square \square \square \square  \square \square z_1 \square &\rightarrow &\square \square \square \square  \square \square z_1 \square &\rightarrow &\square \square \square \square \square \square z_1 \square &\rightarrow & \square \square \square \square \square \square z_e \square\\
		0 z_100 \square &  &z_1000 \square & &z_1 \square 000 \square & &z_e000 \square
	\end{align*}
	\subsubsection*{b)}
	Die Turingmaschine $M$ extrahiert bei der Funktion $f$ alle $0$en aus der Eingabe (also auf Band 1) und schreibt sie auf Band 2. Au"serdem wird die Eingabe mit $\square$ überschrieben.\newpage
	
	\subsubsection*{c)}
	Jetzt soll die 2-Band Turingmaschine $M$ für die Funktion $f$ auf eine 1-Band Turingmaschine $M'$ für dieselbe Funktion übertragen werden.
	\begin{align*}
		\delta \left( z_0,1\right) &= \left( z_0, \square, R\right)\\
		\delta \left( z_0,0\right) &= \left( z_1, 0, R\right)\\
		\delta \left( z_0,\square\right) &= \left( z_e, \square, N\right)\\
		\delta \left( z_1,1\right) &= \left( z_2, 0, L\right)\\
		\delta \left( z_1,0\right) &= \left( z_1, 0, R\right)\\
		\delta \left( z_1,\square\right) &= \left( z_e, \square, N\right)\\
		\delta \left( z_2,0\right) &= \left( z_2, 0, L\right)\\
		\delta \left( z_2,\square\right) &= \left( z_3, \square, R\right)\\
		\delta \left( z_3,0\right) &= \left( z_0, \square, R\right)
	\end{align*}
	
	\section*{Aufgabe 3}
	\subsubsection*{a)}
	
	
\end{document}