\documentclass[a4paper,12pt]{article}
\usepackage{fancyhdr}
\usepackage[ngerman,german]{babel}
\usepackage{german}
\usepackage[utf8]{inputenc}
\usepackage[active]{srcltx}
%\usepackage{algorithm}
%\usepackage[noend]{algorithmic}
\usepackage{amsmath}
\usepackage{amssymb}
\usepackage{amsthm}
\usepackage{bbm}
\usepackage{enumerate}
\usepackage{graphicx}
\usepackage{ifthen}
\usepackage{listings}
\usepackage{struktex}
\usepackage{hyperref}
\usepackage{tabularx}
\usepackage[linesnumbered, ruled]{algorithm2e}
\SetKwRepeat{Do}{do}{while}

\newcommand{\Fach}{Berechenbarkeit und Komplexität}
\newcommand{\Name}{Lukas Harsch}
\newcommand{\Matrikelnummer}{979272}
\newcommand{\Semester}{Sommersemester 2019}
\newcommand{\Kapitel}{4}
\newcommand{\Titel}{}

\setlength{\parindent}{0em}
\topmargin 0cm
\oddsidemargin 0cm
\evensidemargin 0cm
\setlength{\textheight}{9.2in}
\setlength{\textwidth}{6.0in}

\hypersetup{
	pdftitle={\Fach{}: "Ubungsblatt \Kapitel{}},
	pdfauthor={\Name},
	pdfborder={0 0 0}
}

%\lstset{ %
%	language=java,
%	basicstyle=\footnotesize\tt,
%	showtabs=false,
%	tabsize=2,
%	captionpos=b,
%	breaklines=true,
%	extendedchars=true,
%	showstringspaces=false,
%	flexiblecolumns=true,
%}

\title{"Ubungsblatt \Kapitel}
\author{\Name{}}

\begin{document}
	\thispagestyle{fancy}
	\lhead{\sf \small \Name{}}
	\chead{\sf \small \Fach}
	\rhead{\sf \small \Semester{}}
	\rfoot{\sf \tiny Keine Gewähr auf Richtigkeit und Vollständigkeit}
	\lfoot{\sf \tiny CC BY-NC-SA}
	\begin{center}
		\LARGE \sf \textbf{"Ubungsblatt \Kapitel}\\
		\LARGE \sf \Titel
	\end{center}
	\vspace*{0.2cm}
	
	\section*{Aufgabe 1}
	\subsubsection*{a)}
	\begin{algorithm}[H]
		\KwData{$x_1$ Eingabe\;}
		\If{$x_1 = 0$}{GOTO M0\;}
		\If{$x_1 = 1$}{GOTO M0\;}
		$x_4 = \left(x_1 DIV 2\right) + 1$\;
		\For{$x_2 = 2$ \KwTo $x_4$}{
			\For{$x_3 = x_2$ \KwTo $x_4$}{
			\If{$x_1 = x_2 + x_3$}{GOTO M0\;}
			}
		}
		$x_0 = 1$\;
		HALT\;
		\caption{Funktion f: Primzahltest}
	\end{algorithm}
	
	\begin{algorithm}{H}
		$x_0 = 0$\;
		HALT\;
		\caption{M0}
	\end{algorithm}
	
	Achtung: \textbf{for}-Schleifen sind hier eigentlich nicht erlaubt, können aber in einem Hilfsprogramm mit \textbf{if}-Anweisungen konstruiert werden.
	
	\subsubsection*{b)}
	\begin{algorithm}[H]
		$x_0 = h\left(x_2\right)$ \tcp{n}
		$x_1 = h\left(x_3\right)$ \tcp{m}
		\caption{Funktion g: "`erweiterter"' Primzahltest}
	\end{algorithm}

	\vspace{1cm}

	\begin{algorithm}[H]
		\KwData{$x_1 =$ Eingabe\;}
		$x_2 = 0$\;
		$x_3 = 2$\;
		\caption{Funktion h}
	\end{algorithm}

	\begin{algorithm}[H]
		\If{$x_2 = x_1$}{GOTO M3\;}
		$x_3 = x_3 + 1$
		\caption{M1}
	\end{algorithm}

	\begin{algorithm}[H]
		\eIf{$f \left(x_3\right) = 1$}{
			$x_2 = x_2+1$\;
			GOTO M1\;
		}{
			$x_3 = x_3 + 1$\;
			GOTO M2\;
		}
		\caption{M2}
	\end{algorithm}

	\begin{algorithm}[H]
		$x_0 = x_3$\;
		HALT\;
		\caption{M3}
	\end{algorithm}
	\section*{Aufgabe 2}
	\subsubsection*{a)}
	\begin{align*}
		fak\left(1\right)& = 1\\
		fak\left(n\right)& = mult\left(n, fak\left(n-1\right)\right)
	\end{align*}
	\begin{center}
		$\Rightarrow$ Verknüpfungen primitiv rekursiver Funktionen sind ebenfalls primitiv rekursiv
	\end{center}

	\subsubsection*{b)}
	\begin{align*}
		abs\left(n,m\right)& = sub\left(n,m\right) + sub\left(m,n\right)
	\end{align*}
	\begin{center}
		$\Rightarrow$ $sub$ ist primitiv rekursiver, damit ist $abs$ ebenfalls primitiv rekursiv
	\end{center}
	\newpage
	\subsubsection*{c)}
	\begin{align*}
		max\left(n,m\right)& = add\left(sub\left(n,m\right),m\right)
	\end{align*}
	\begin{center}
		$\Rightarrow$ $add$ und $sub$ sind primitiv rekursiv, damit auch $max$
	\end{center}

	\section*{Aufgabe 3}
	\subsubsection*{a)}
	\begin{enumerate}[i)]
		\item \textbf{IA:} $a\left(1,0\right) = a\left(0,1\right) = 2$\\
			\textbf{IH:} $a\left(1,n\right) = n + 2$\\
			\textbf{IS:} $a\left(1, n+1\right) = a\left(0,a\left(1,n\right)\right) = a\left(1,n\right)+1 \overset{\text{IH}}{=} n+3$ \hfill $\square$
		\item \textbf{IA:} $a\left(2,0\right) = a\left(1,1\right) = a\left(0,2\right) = 3$\\
			\textbf{IH:} $a\left(2,y\right) = 2y+3$\\
			\textbf{IS:} $a\left(2, y+1\right) = a\left(1,a\left(2,y\right)\right) = 0$\\
			\hspace*{0.55cm} $a\left(1, 2y+3\right) = 2y+5 = 2\left(y+1\right)+3$ \hfill $\square$
	\end{enumerate}

	\subsubsection*{b)}
	Zu Zeigen: $f_n$ LOOP-berechenbar, $\mathcal{O}\left(n\right)$ LOOP-Schleifen
	\begin{align*}
		\text{\textbf{IA: }} &n = 1\\
		&f_n\left(k\right) = \underbrace{a\left(1,k\right) = k+2}_{\text{LOOP berechenbar}}\\
		\text{\textbf{IH: }} &f_n \text{ LOOP-berechenbar mit } \mathcal{O}\left(n\right) \text{LOOP-Schleifen}\\
		&\text{für ein beliebiges, aber festes } n \in \mathbb{N}\\
		\text{\textbf{IS: }} &n \rightarrow n+1\\
		&f_{n+1}\left(k\right) = a\left(n+1,k\right)\\
		&\overset{k \neq 0}{=} f_n\left(n,a\left(n+1, k-1\right)\right)\\
		&= f_n\left(a\left(n+1, k-1\right)\right) = f_n\left(f_{n+1}\left(k-1\right)\right)\\
		&=\left(f_n^k\right)\left(a\left(n+1,0\right)\right)\\
		&=\left(f_n^k\right)\left(a\left(n,1\right)\right)\\
		&=\left(f_n^k\right)\left(f_n\left(1\right)\right)\\
		&=f_n^{k+1}\left(1\right)
	\end{align*}
	\begin{center}
		$\Rightarrow$ Somit ist $f_{n+1}$ mit $\mathcal{O}\left(n+1\right)$ LOOP berechenbar
	\end{center}
 	\hfill $\square$
	
\end{document}