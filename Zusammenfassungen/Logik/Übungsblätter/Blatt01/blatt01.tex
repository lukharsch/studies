\documentclass[a4paper,12pt]{article}
\usepackage{fancyhdr}
\usepackage[ngerman,german]{babel}
\usepackage{german}
\usepackage[utf8]{inputenc}
\usepackage[active]{srcltx}
\usepackage{algorithm}
\usepackage[noend]{algorithmic}
\usepackage{amsmath}
\usepackage{amssymb}
\usepackage{amsthm}
\usepackage{bbm}
\usepackage{enumerate}
\usepackage{graphicx}
\usepackage{ifthen}
\usepackage{listings}
\usepackage{struktex}
\usepackage{hyperref}
\usepackage{tabularx}

\newcommand{\Fach}{Logik}
\newcommand{\Name}{Lukas Harsch}
\newcommand{\Matrikelnummer}{979272}
\newcommand{\Semester}{Sommersemester 2019}
\newcommand{\Kapitel}{1}
\newcommand{\Titel}{}

\setlength{\parindent}{0em}
\topmargin 0cm
\oddsidemargin 0cm
\evensidemargin 0cm
\setlength{\textheight}{9.2in}
\setlength{\textwidth}{6.0in}

\hypersetup{
	pdftitle={\Fach{}: Blatt \Kapitel{}},
	pdfauthor={\Name},
	pdfborder={0 0 0}
}

\lstset{ %
	language=java,
	basicstyle=\footnotesize\tt,
	showtabs=false,
	tabsize=2,
	captionpos=b,
	breaklines=true,
	extendedchars=true,
	showstringspaces=false,
	flexiblecolumns=true,
}

\title{Kapitel \Kapitel}
\author{\Name{}}

\begin{document}
	\thispagestyle{fancy}
	\lhead{\sf \small \Name{}}
	\chead{\sf \small \Fach}
	\rhead{\sf \small \Semester{}}
	\rfoot{\sf \tiny Keine Gewähr auf Richtigkeit und Vollständigkeit}
	\lfoot{\sf \tiny CC BY-NC-SA}
	\begin{center}
		\LARGE \sf \textbf{"Ubungsblatt \Kapitel}\\
		\LARGE \sf \Titel
	\end{center}
	\vspace*{0.2cm}
	
	\section*{Aufgabe 1.1}
	\subsubsection*{a)}
	\begin{center}
		$B \vee W$\\
		mit $B$ := Bier und $W$ := Wein
	\end{center}
	\subsubsection*{b)}
	\begin{center}
		$B \oplus W$\\
		Aussage: "`entweder oder"', statt "`oder"'
	\end{center}
	\subsubsection*{c)}
	\begin{center}
		$\bigvee \limits_{i = 1} ^{n} \left( A_i \wedge \bigwedge \limits_{j = 1, j \neq i} ^{n} \overline{A_j} \right)$
	\end{center}
	\begin{align*}
		= &\left( A_1  \wedge \overline{A_2} \wedge \cdots \wedge \overline{A_n} \right) \vee \\
		&\left( \overline{A_1}  \wedge A_2 \wedge \cdots \wedge \overline{A_n} \right) \vee \\
		&\vdots \\
		&\left( \overline{A_1}  \wedge \overline{A_2} \wedge \cdots \wedge A_n \right)
	\end{align*}
	
	\section*{Aufgabe 1.2}
	\subsubsection*{a)}
	\begin{enumerate}
		\item $\mathcal{A} \left( A_1\right) = \mathcal{A} \left( A_2\right) = \cdots = \mathcal{A} \left( A_n\right) = 1$
		\item $\mathcal{A} \left( A_1\right) = \mathcal{A} \left( A_2\right) = \cdots = \mathcal{A} \left( A_n\right) = 0$
	\end{enumerate}
	\subsubsection*{b)}
	\hspace{0.3cm} $\mathcal{A} \left( A_1\right) = 1, \mathcal{A} \left( A_2\right) = \cdots = \mathcal{A} \left( A_n\right) = 0$
	\newpage

	\section*{Aufgabe 1.3}
	\subsubsection*{a)}
	\begin{center}
		$F_1 = \underbrace{\left( \left( A \rightarrow B\right) \wedge \left( B \rightarrow A \right) \right)}_{:= X} \vee C$
	\end{center}
	\begin{table}[H]
		\centering
		\begin{tabular}{c|c|c|c|c|c|c}
			$C$ & $A$ & $B$ & $\left( A \rightarrow B\right)$ & $\left( B \rightarrow A\right)$ & $X$ & $F$\\ \hline
			0 & 0 & 0 & 1 & 1 & 1 & 1 \\
			0 & 0 & 1 & 1 & 0 & 0 & 0 \\
			0 & 1 & 0 & 0 & 1 & 0 & 0 \\
			0 & 1 & 1 & 1 & 1 & 1 & 1 \\
			1 & 0 & 0 & 1 & 1 & 1 & 1 \\
			1 & 0 & 1 & 1 & 0 & 0 & 1 \\
			1 & 1 & 0 & 0 & 1 & 0 & 1 \\
			1 & 1 & 1 & 1 & 1 & 1 & 1
		\end{tabular}
	\end{table}
	\begin{center}
		$\Rightarrow$ $F_1$ ist erfüllbar, weil mindestens eine 1 in der $F$-Spalte steht, aber keine Tautologie (also nicht gültig), weil es auch Belegungen für $F$ gibt, die kein Modell sind.
	\end{center}

	\subsubsection*{b)}
	\begin{center}
		$F_2 = \left( B \rightarrow A \right) \leftrightarrow \underbrace{\left( A \rightarrow \left( A \wedge B \right) \right) \wedge \left( A \oplus B \right)}_{:= Y}$
	\end{center}
	\begin{table}[H]
		\centering
		\begin{tabular}{c|c|c|c|c|c|c|c}
			$A$ & $B$ & $\left( B \rightarrow A\right)$ & $\left( A \wedge B\right)$ & $\left( A \rightarrow \left( A \wedge B\right)\right)$ & $A \oplus B$ & $Y$ & $F$\\ \hline
			0 & 0 & 1 & 0 & 1 & 0 & 0 & 0 \\
			0 & 1 & 0 & 0 & 1 & 1 & 1 & 0 \\
			1 & 0 & 1 & 0 & 0 & 1 & 0 & 0 \\
			1 & 1 & 1 & 1 & 1 & 0 & 0 & 0
		\end{tabular}
	\end{table}
	\begin{center}
		$\Rightarrow$ $F_2$ ist nicht erfüllbar, da in der $F$-Spalte nur 0en stehen und damit keine Tautologie, also nicht gültig.
	\end{center}
	\newpage
	
	\section*{Aufgabe 1.4}
	\begin{center}
		\textbf{Jedes Element $i$ ist in mindestens einem Kasten $j$:}\\
		$F = \bigwedge \limits_{i = 1}^{2n+1} \bigvee \limits_{j = 1} ^{n} A_{i,j}$
	\end{center}
	\begin{center}
		\textbf{In jedem Kasten $j$ befinden sich mindestens 2 Dinge $i$ und $i'$:}\\
		$G = \bigwedge \limits_{j = 1}^{n} \bigvee \limits_{i = 1} ^{2n} \bigvee \limits_{i' = i+1} ^{2n+1} \left( A_{i,j} \wedge A_{i',j} \right)$
	\end{center}
	\begin{center}
		\textbf{In jedem Kasten $j$ befinden sich höchstens 3 Dinge $i$, $i'$ und $i''$:}
		$H = \bigwedge \limits_{j = 1}^{n} \bigwedge \limits_{i = 1} ^{2n-2} \bigwedge \limits_{i' = i+1} ^{2n-1} \bigwedge \limits_{i'' = i'+1} ^{2n} \bigwedge \limits_{i''' = i''+1} ^{2n+1} \left( A_{i,j} \wedge A_{i',j} \wedge A_{i'',j} \wedge A_{i''',j}\right)$
	\end{center}
	\begin{center}
		\textbf{Jedes Element $i$ ist in höchstens einem Fach $j$ oder $j'$:}
		$I = \bigwedge \limits_{i = 1}^{2n+1} \bigwedge \limits_{j = 1} ^{n-1} \bigwedge \limits_{j'=j+1} ^{n} \lnot \left( A_{i,j} \wedge A_{i,j'} \right)$
	\end{center}
	\begin{center}
		$\Rightarrow$ Also ist die zu suchende Formel $A = \left( F \wedge G \wedge H \wedge I \right)$
	\end{center}
	
	\section*{Aufgabe 1.5}
	\begin{center}
		$F = A \wedge B \wedge \left( \overline{G} \vee D\right) \wedge \overline{C} \wedge \left( \overline{A} \vee \overline{B} \vee \overline{D} \vee E\right) \wedge G \wedge \left( \overline{E} \vee C\right)$\\
	\end{center}
	Diese Formel wird nun so umgewandelt, dass sie nur aus Implikationen besteht:
	\begin{align*}
		F' = &\left( 1 \rightarrow A\right) \wedge \left( 1 \rightarrow B\right) \wedge \left( G \rightarrow D\right) \wedge \left( C \rightarrow 0\right) \wedge \\&\left( \left(A \wedge B \wedge D\right) \rightarrow E\right) \wedge \left( 1 \rightarrow G\right) \wedge \left( E \rightarrow C\right)
	\end{align*}
	Auf diese Formel $F'$ wird nun der Markierungsalgorithmus angewandt:
	
\end{document}