\documentclass[a4paper,12pt]{article}
\usepackage{fancyhdr}
\usepackage[ngerman,german]{babel}
\usepackage{german}
\usepackage[utf8]{inputenc}
\usepackage[active]{srcltx}
\usepackage{algorithm}
\usepackage[noend]{algorithmic}
\usepackage{amsmath}
\usepackage{amssymb}
\usepackage{amsthm}
\usepackage{bbm}
\usepackage{enumerate}
\usepackage{graphicx}
\usepackage{ifthen}
\usepackage{listings}
\usepackage{struktex}
\usepackage{hyperref}
\usepackage{tabularx}
\usepackage[noeepic]{qtree}

\newcommand{\Fach}{Logik}
\newcommand{\Name}{Lukas Harsch}
\newcommand{\Matrikelnummer}{979272}
\newcommand{\Semester}{Sommersemester 2019}
\newcommand{\Kapitel}{3}
\newcommand{\Titel}{}

\setlength{\parindent}{0em}
\topmargin 0cm
\oddsidemargin 0cm
\evensidemargin 0cm
\setlength{\textheight}{9.2in}
\setlength{\textwidth}{6.0in}

\hypersetup{
	pdftitle={\Fach{}: "Ubungsblatt \Kapitel{}},
	pdfauthor={\Name},
	pdfborder={0 0 0}
}

\lstset{ %
	language=java,
	basicstyle=\footnotesize\tt,
	showtabs=false,
	tabsize=2,
	captionpos=b,
	breaklines=true,
	extendedchars=true,
	showstringspaces=false,
	flexiblecolumns=true,
}

\title{"Ubungsblatt \Kapitel}
\author{\Name{}}

\begin{document}
	\thispagestyle{fancy}
	\lhead{\sf \small \Name{}}
	\chead{\sf \small \Fach}
	\rhead{\sf \small \Semester{}}
	\rfoot{\sf \tiny Keine Gewähr auf Richtigkeit und Vollständigkeit}
	\lfoot{\sf \tiny CC BY-NC-SA}
	\begin{center}
		\LARGE \sf \textbf{"Ubungsblatt \Kapitel}\\
		\LARGE \sf \Titel
	\end{center}
	\vspace*{0.2cm}
	
	\section*{Aufgabe 3.1}
	"Ahnlich zum Beweis der Widerlegungsvollständigkeit im Skript:\\
	
	Beweis durch Induktion über die Anzahl $n$ der in Formel $F$ vorkommenden atomaren Formeln.\\
	
	\underline{IA:}\\
	Falls $n = 0$, so kann nur $F = \left\lbrace \square\right\rbrace$ sein und wir sind fertig\\
	\underline{IH:}\\
	Für eine unerfüllbare Formel $F$ mit $n$ atomaren Formeln gibt es einen Unerfüllbarkeitsbeweis, der die Regeln der P-Resolution folgt\\
	\underline{IS:}\\
	Betrachte eine unerfüllbare Formel $F$ mit $n+1$ atomaren Formeln.\\
	Erzeuge $F_0$ und $F_1$ wie folgt:\\
	$F_0$ entsteht aus $F$, indem jedes Vorkommen von $A_{n+1}$ in einer Klausel gestrichen wird. Dies entspricht dem Belegen von $A_{n+1}$ mit 0.\\
	Analog entsteht $F_1$ aus $F$, also als Belegung von $A_{n+1}$ mit 1.\\
	Da $F$ unerfüllbar ist, müssen auch $F_0$ und $F_1$ unerfüllbar sein, ansonsten könnte aus deren erfüllender Belegung eine erfüllende Belegung für $F$ konstruiert werden.\\

	Da $A_{n+1}$ in $F_0$ und $F_1$ nicht mehr vorkommt, haben $F_0$ und $F_1$ nur noch $n$ atomare Formeln und wir können die \underline{IH} anwenden und erhalten einen Unerfüllbarkeitsbeweis mittels P-Resolution für $F_0 \left(K_0^1, K_0^2, K_0^3, \cdots, K_0^m\right)$ und $F_1 \left(K_1^1, K_1^2, K_1^3, \cdots, K_1^l\right)$, mit $K_0^m = K_1^l = \square$.\\
	Fügen wir nun die atomaren Formeln wieder in die Klauseln der Unerfüllbarkeitsbeweise von Formel $F_0$ und $F_1$ ein, können wir zwischen zwei Fällen unterscheiden:
	\begin{enumerate}[1.]
		\item Trotz Wiedereinfügen von $A_{n+1}$ führen die gleichen Resolutionsschritte des Unerfüllbarkeitsbeweises von $F_0$ zur leeren Klausel.\\
		Dann sind wir fertig.
		\item Folgen des Beweises von $F_0$ nach Wiedereinfügen von $A_{n+1}$ führt nicht zur leeren Klausel. Dann muss $K^m = \left\lbrace A_{n+1}\right\rbrace$ sein. Resolviert man $K^m$ mit jeder Klausel aus $F$, entstehen die Klauseln, die in $F_1$ vorkommen.\\
		Folgen wir nun dem Beweis von $F_1$, erhalten wir die leere Klausel.
	\end{enumerate}
	\hfill $\square$
	\newpage
	\underline{Beispiel:} Induktionsschritt mit $n+1 = 4$
	\begin{align*}
		&F = \left\lbrace \left\lbrace A_1\right\rbrace, \left\lbrace \overline{A_2}, A_4\right\rbrace, \left\lbrace \overline{A_1}, A_2, A_4\right\rbrace, \left\lbrace A_3, \overline{A_4}\right\rbrace, \left\lbrace \overline{A_1}, \overline{A_3}, \overline{A_4}\right\rbrace\right\rbrace
	\end{align*}
	\begin{align*}
		&F = \left\lbrace \left\lbrace A_1\right\rbrace, \left\lbrace \overline{A_2}\right\rbrace, \left\lbrace \overline{A_1}, A_2\right\rbrace\right\rbrace &\left( A_n \text{ wird mit 0 belegt}\right)\\
		&F = \left\lbrace \left\lbrace A_1\right\rbrace, \left\lbrace A_3\right\rbrace, \left\lbrace \overline{A_1}, \overline{A_3}\right\rbrace\right\rbrace &\left( A_n \text{ wird mit 1 belegt}\right)\\
	\end{align*}
	Baum siehe Bild, TODO
	
	\section*{Aufgabe 3.2}
	\subsubsection*{a)}
	Zeige die Aussage, indem gezeigt wird, dass höchstens $4^n$ Klauseln existieren können.\\
	
	Betrachtet man eine Klausel, kann eine Atomformel auf genau vier verschiedene Arten darin auftreten: gar nicht, positiv, negativ oder positiv \textit{und} negativ. Diese vier Optionen gibt es für jede Atomformel, die Kombination aller dieser Optionen gibt also alle möglichen Klauseln, was $4^n$ sind. 
	
	\subsubsection*{b)}
	Betrachten wir die Folge $F = Res^0\left(F\right)$, $Res^1\left(F\right)$, $\cdots$, $Res^k\left(F\right) = Res^*\left(F\right)$.\\
	
	Im schlimmsten Fall unterscheidet sich $Res^i\left(F\right)$ und $Res^{i+1}\left(F\right)$ nur durch genau eine Klausel, also $\left|Res^0\left(F\right)\right| - \left|Res^0\left(F\right)\right| = 1$ für alle $i$. Da $\left|Res^0\left(F\right)\right|$ und wir aus a) wissen, dass $\left|Res^*\left(F\right)\right| \leq 4^n$, wissen wir nun auch, dass nach spätestens $4^n$ Schritten alle möglichen Klauseln entstanden sind.\\
	
	Also ist $Res^k\left(F\right) = Res^{4^n}\left(F\right) = Res^*\left(F\right)$.\\
	(In den meisten Fällen geschieht das aber schon viel früher)
	
	\section*{Aufgabe 3.3}
	Aus der Definition von Resolution folgt:\\
	
	$F$ ist unerfüllbar $\Rightarrow$ $F$ lässt sich zu $\square$ resolvieren.\\
	
	Somit kann der Algorithmus Unerfüllbarkeit erkennen (weil wir aus Aufgabe 1 wissen, dass es nur maximal $4^n$ Klauseln geben kann um überhaupt erst resolvieren zu können).\\
	Wenn $F$ nach einer gescheiterten Resolution \textit{nicht} $\square$ ausgibt, so ist $F$ erfüllbar. 
	
	
	\section*{Aufgabe 3.4}
	\begin{center}
		$F = \forall x \forall y \forall z \left(\left(E\left(x,y\right) \wedge E\left(y,z\right)\right) \rightarrow \overline{E\left(x,z\right)}\right)$
	\end{center}
	\subsubsection*{a)}
	$\mathcal{A}$ ist ein Modell für $F$. Argumentation durch Widerspruch:\\
	
	Angenommen $\mathcal{A}$ sei kein Modell. Dann gäbe es eine Belegung der Variablen, sodass $F^{\mathcal{A}}$ zu 0 ausgewertet wird.\\
	Dies kann nur der Fall sein, wenn die linke Seite der Implikation zu 1 ausgewertet wird und die rechte Seite zu 0.\\
	Die Grundmenge sind die natürlichen Zahlen, daher wissen wir, dass die Summe zweier Zahlen ungerade ist, genau dann wenn eine der beiden Zahlen gerade und die andere ungerade ist.
	\begin{enumerate}[1.]
		\item $x$ ist gerade: Damit der linke Teil der Implikation zu 1 ausgewertet wird, muss $y$ ungerade sein und dadurch $z$ gerade.\\
		Da aber $x$ und $z$ gerade $\Rightarrow$ rechte Seite der Implikation 1.
		\item $x$ ist ungerade: (Analog zu a)) 
	\end{enumerate}

	\subsubsection*{b)}
	$\mathcal{B}$ ist kein Modell.\\
	
	Betrachte dazu folgenden Fall: $\left(x=0, y=0, z=0\right)$
	

\end{document}