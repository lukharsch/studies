\documentclass[a4paper,12pt]{article}
\usepackage{fancyhdr}
\usepackage[ngerman,german]{babel}
\usepackage{german}
\usepackage[utf8]{inputenc}
\usepackage[active]{srcltx}
\usepackage{algorithm}
\usepackage[noend]{algorithmic}
\usepackage{amsmath}
\usepackage{amssymb}
\usepackage{amsthm}
\usepackage{bbm}
\usepackage{enumerate}
\usepackage{graphicx}
\usepackage{ifthen}
\usepackage{listings}
\usepackage{struktex}
\usepackage{hyperref}
\usepackage{tabularx}
\usepackage[noeepic]{qtree}

\newcommand{\Fach}{Logik}
\newcommand{\Name}{Lukas Harsch}
\newcommand{\Matrikelnummer}{979272}
\newcommand{\Semester}{Sommersemester 2019}
\newcommand{\Kapitel}{2}
\newcommand{\Titel}{}

\setlength{\parindent}{0em}
\topmargin 0cm
\oddsidemargin 0cm
\evensidemargin 0cm
\setlength{\textheight}{9.2in}
\setlength{\textwidth}{6.0in}

\hypersetup{
	pdftitle={\Fach{}: "Ubungsblatt \Kapitel{}},
	pdfauthor={\Name},
	pdfborder={0 0 0}
}

\lstset{ %
	language=java,
	basicstyle=\footnotesize\tt,
	showtabs=false,
	tabsize=2,
	captionpos=b,
	breaklines=true,
	extendedchars=true,
	showstringspaces=false,
	flexiblecolumns=true,
}

\title{"Ubungsblatt \Kapitel}
\author{\Name{}}

\begin{document}
	\thispagestyle{fancy}
	\lhead{\sf \small \Name{}}
	\chead{\sf \small \Fach}
	\rhead{\sf \small \Semester{}}
	\rfoot{\sf \tiny Keine Gewähr auf Richtigkeit und Vollständigkeit}
	\lfoot{\sf \tiny CC BY-NC-SA}
	\begin{center}
		\LARGE \sf \textbf{"Ubungsblatt \Kapitel}\\
		\LARGE \sf \Titel
	\end{center}
	\vspace*{0.2cm}
	
	\section*{Aufgabe 2.1}
	\subsubsection*{a)}
	Jede Hornformel, in der keine Faktenklauseln vorkommen, ist erfüllbar.\\
	\begin{center}
		\underline{WAHR}, denn dann kann die Hornformel nur noch aus definiten Hornklauseln, also z.B. $\left(A \rightarrow B\right)$ oder Zielklauseln, z.B. $\left(\lnot C\right)$ bestehen.\\
		Nun lässt sich eine Belegung finden, die bei jeder solcher Hornformel ein Model ist, nämlich wenn alle Literale auf 0 gesetzt werden.
	\end{center}
	
	\subsubsection*{b)}
	Jede Hornformel, in der keine Zielklauseln vorkommen, ist erfüllbar.\\
	\begin{center}
		\underline{WAHR}, denn die Hornformel kann dann nur noch aus Faktenklauseln, also z.B. $A$ und definiten Klauseln bestehen.\\
		Auch hier lässt sich eine Belegung finden, die bei jeder solcher Hornformel Model ist, nämlich wenn alle Literale auf 1 gesetzt werden.
	\end{center}
	
	\subsubsection*{c)}
	Bei der Resolution zweier Hornklauseln kann eine Klausel entstehen, die keine Hornklausel ist.\\
	\begin{center}
		\underline{FALSCH} - Hornklauseln haben maximal ein positives Literal. Da bei der Resolution in einer der beiden Klauseln maximal ein positives Literal verschwindet, kann die resultierende Klausel wieder maximal ein positives Literal haben.
	\end{center}
 	\hfill $\square$
 	
 	\section*{Aufgabe 2.4}
 	\subsubsection*{a)}
 	\begin{enumerate}[1.]
 		\item $B \vee W$ \hfill $B$: Bier, $W$: Wein, $C$: Cola
 		\item $\lnot W \vee C$
 		\item $B \vee W \vee C$
 	\end{enumerate}
 
 	\subsubsection*{b)}
 	\Tree [.B, W [.B [.C eins ] [.D zwei ] ].B [.E {3 und 4} ] ].B, W
 	
\end{document}