\documentclass[a4paper,12pt]{article}
\usepackage{fancyhdr}
\usepackage[ngerman,german]{babel}
\usepackage{german}
\usepackage[utf8]{inputenc}
\usepackage[active]{srcltx}
\usepackage{algorithm}
\usepackage[noend]{algorithmic}
\usepackage{amsmath}
\usepackage{amssymb}
\usepackage{amsthm}
\usepackage{bbm}
\usepackage{enumerate}
\usepackage{graphicx}
\usepackage{ifthen}
\usepackage{listings}
\usepackage{struktex}
\usepackage{hyperref}
\usepackage{tabularx}
\usepackage[framemethod=tikz]{mdframed}
\usepackage{extarrows}
\usepackage{xcolor}
\definecolor{darkblue}{RGB}{0,0,102}
\definecolor{darkred}{RGB}{153,0,0}
\definecolor{darkgreen}{RGB}{0,102,0}

\newcommand{\Fach}{Analysis I}
\newcommand{\Name}{Lukas Harsch}
\newcommand{\Matrikelnummer}{979272}
\newcommand{\Semester}{Sommersemester 2019}
\newcommand{\Kapitel}{3}
\newcommand{\Titel}{}

% Task
\mdtheorem[
linecolor=darkblue,
frametitlefont=\sffamily\bfseries\color{white},
frametitlebackgroundcolor=darkblue,
]{bsp}{Beispiel}[section]

%Tipp
\mdtheorem[
linecolor=blue,
frametitlefont=\sffamily\bfseries\color{white},
frametitlebackgroundcolor=darkred,
]{tipp}{Hinweis}[section]

%Definition
\mdtheorem[
linecolor=darkgreen,
frametitlefont=\sffamily\bfseries\color{white},
frametitlebackgroundcolor=darkgreen,
]{defi}{Definition}[section]

\setlength{\parindent}{0em}
\topmargin 0cm
\oddsidemargin 0cm
\evensidemargin 0cm
\setlength{\textheight}{9.2in}
\setlength{\textwidth}{6.0in}

\hypersetup{
	pdftitle={\Fach{}: Tutorium \Kapitel{}},
	pdfauthor={\Name},
	pdfborder={0 0 0}
}

\lstset{ %
	language=java,
	basicstyle=\footnotesize\tt,
	showtabs=false,
	tabsize=2,
	captionpos=b,
	breaklines=true,
	extendedchars=true,
	showstringspaces=false,
	flexiblecolumns=true,
}

\title{Tutorium \Kapitel}
\author{\Name{}}

\begin{document}
	\thispagestyle{fancy}
	\lhead{\sf \small \Name{}}
	\chead{\sf \small \Fach}
	\rhead{\sf \small \Semester{}}
	\rfoot{\sf \tiny Keine Gewähr auf Richtigkeit und Vollständigkeit}
	\lfoot{\sf \tiny CC BY-NC-SA}
	\begin{center}
		\LARGE \sf \textbf{Tutorium \Kapitel}\\
		\LARGE \sf \Titel
	\end{center}
	\vspace*{0.2cm}
	
	\section{Häufungspunkte}
	\begin{bsp}
		Bestimme die Häufungspunkte von folgender Folge:
		\begin{center}
			$a_n = \frac{1}{2^n} + \left(-1\right)^n$, $n \in \mathbb{N}_0$
		\end{center}
	\end{bsp}
	Bestimme Teilfolgen $a_{2n}$ und $a_{2n+1}$
	\begin{align*}
		a_{2n} &= \frac{1}{2^{2n}} + \left(-1\right)^{2n} = \frac{1}{4^n} + 1 \xrightarrow{n\rightarrow\infty} 1 \Rightarrow \left(a_n\right)_{n \in \mathbb{N}_0} \text{hat HP} 1 \Rightarrow \limsup\\
		a_{2n+1} &= \frac{1}{2^{2n+1}} + \left(-1\right)^{2n+1} = \frac{1}{2^{2n} \cdot 2} - 1 = \frac{1}{4^n \cdot 2} -1 \xrightarrow{n\rightarrow\infty} -1 \\
		&\Rightarrow \left(a_n\right)_{n \in \mathbb{N}_0} \text{hat HP} -1 \Rightarrow \liminf
	\end{align*}
	
	\section{Cauchyfolgen}
	\begin{defi}
		Eine Folge $\left(a_n\right)_{n \in \mathbb{N}_0}$ hei"st Cauchyfolge genau dann, wenn
		\begin{center}
			$\forall \epsilon > 0$ $\exists n_0 \in \mathbb{N}$ $\forall n,m \geq n_0 : \left|a_n - a_m \right|$
		\end{center}
	\end{defi}
	\begin{bsp}
		Ist die Folge $a_n$ Cauchyfolge?
		\begin{center}
			$a_n = \frac{1}{n^2+n}$
		\end{center}
	\end{bsp}
	Es seien $n,m \in \mathbb{N}$ mit $n, m \geq n_0$ und $\epsilon > 0$ beliebig. O.B.d.A $n \geq m$\\
	\begin{align*}
		\left|a_m - a_n\right| &= \left|\frac{1}{m^2+m} - \frac{1}{n^2+n}\right| = \left|\frac{n^2+n-m^2-m}{\left(n^2+n\right) \left(m^2+m\right)}\right| \leq \left|\frac{n^2+n}{\left(n^2+n\right) \left(m^2+m\right)}\right|\\
		&\left|\frac{1}{m^2+m}\right| = \frac{1}{m^2+m} \leq \frac{1}{m} < \epsilon \Leftrightarrow \frac{1}{\epsilon} < m\\
		&\Rightarrow \text{Setze } n_0 = \left\lceil\frac{1}{\epsilon}\right\rceil + 1
	\end{align*}
	\begin{bsp}
		Ist die Folge $a_n$ Cauchyfolge?
		\begin{center}
			$a_n = \frac{7n^3-13n^2}{n^3} = 7 - \frac{13}{n}$
		\end{center}
	\end{bsp}
	Es seien $n,m \in \mathbb{N}$ mit $n, m \geq n_0$ und $\epsilon > 0$ beliebig. O.B.d.A $n \geq m$\\
	\begin{align*}
		\left|a_n - a_m\right| &= \left|7-\frac{13}{n}-\left(7-\frac{13}{m}\right)\right| = \left|\frac{13}{m} - \frac{13}{n}\right| \leq \left|\frac{13}{m} + \frac{13}{n}\right|\\
		&\overset{\triangle\text{-Ungl.}}{\leq} \left|\frac{13}{m}\right| + \left|\frac{13}{n}\right| \leq \frac{13}{m} + \frac{13}{m} = \frac{26}{m} < \epsilon \Leftrightarrow \frac{26}{\epsilon} < m\\
		&\Rightarrow \text{Setze } n_0 = \left\lceil\frac{26}{\epsilon}\right\rceil + 1
	\end{align*}
\end{document}