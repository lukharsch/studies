\documentclass[a4paper,12pt]{article}
\usepackage{fancyhdr}
\usepackage[ngerman,german]{babel}
\usepackage{german}
\usepackage[utf8]{inputenc}
\usepackage[active]{srcltx}
\usepackage{algorithm}
\usepackage[noend]{algorithmic}
\usepackage{amsmath}
\usepackage{amssymb}
\usepackage{amsthm}
\usepackage{bbm}
\usepackage{enumerate}
\usepackage{graphicx}
\usepackage{ifthen}
\usepackage{listings}
\usepackage{struktex}
\usepackage{hyperref}
\usepackage{tabularx}
\usepackage[framemethod=tikz]{mdframed}
\usepackage{extarrows}
\usepackage{xcolor}
\definecolor{darkblue}{RGB}{0,0,102}
\definecolor{darkred}{RGB}{153,0,0}
\definecolor{darkgreen}{RGB}{0,102,0}

\newcommand{\Fach}{Analysis I}
\newcommand{\Name}{Lukas Harsch}
\newcommand{\Matrikelnummer}{979272}
\newcommand{\Semester}{Sommersemester 2019}
\newcommand{\Kapitel}{3}
\newcommand{\Titel}{}

% Task
\mdtheorem[
linecolor=darkblue,
frametitlefont=\sffamily\bfseries\color{white},
frametitlebackgroundcolor=darkblue,
]{bsp}{Beispiel}[section]

%Tipp
\mdtheorem[
linecolor=blue,
frametitlefont=\sffamily\bfseries\color{white},
frametitlebackgroundcolor=darkred,
]{tipp}{Hinweis}[section]

%Definition
\mdtheorem[
linecolor=darkgreen,
frametitlefont=\sffamily\bfseries\color{white},
frametitlebackgroundcolor=darkgreen,
]{defi}{Definition}[section]

\setlength{\parindent}{0em}
\topmargin 0cm
\oddsidemargin 0cm
\evensidemargin 0cm
\setlength{\textheight}{9.2in}
\setlength{\textwidth}{6.0in}

\hypersetup{
	pdftitle={\Fach{}: Tutorium \Kapitel{}},
	pdfauthor={\Name},
	pdfborder={0 0 0}
}

\lstset{ %
	language=java,
	basicstyle=\footnotesize\tt,
	showtabs=false,
	tabsize=2,
	captionpos=b,
	breaklines=true,
	extendedchars=true,
	showstringspaces=false,
	flexiblecolumns=true,
}

\title{Tutorium \Kapitel}
\author{\Name{}}

\begin{document}
	\thispagestyle{fancy}
	\lhead{\sf \small \Name{}}
	\chead{\sf \small \Fach}
	\rhead{\sf \small \Semester{}}
	\rfoot{\sf \tiny Keine Gewähr auf Richtigkeit und Vollständigkeit}
	\lfoot{\sf \tiny CC BY-NC-SA}
	\begin{center}
		\LARGE \sf \textbf{Tutorium \Kapitel}\\
		\LARGE \sf \Titel
	\end{center}
	\vspace*{0.2cm}
	
	\section{Häufungspunkte}
	\begin{bsp}
		Bestimme die Häufungspunkte von folgender Folge:
		\begin{center}
			$a_n = \frac{1}{2^n} + \left(-1\right)^n$, $n \in \mathbb{N}_0$
		\end{center}
	\end{bsp}
	Bestimme Teilfolgen $a_{2n}$ und $a_{2n+1}$
	\begin{align*}
		a_{2n} &= \frac{1}{2^{2n}} + \left(-1\right)^{2n} = \frac{1}{4^n} + 1 \xrightarrow{n\rightarrow\infty} 1 \Rightarrow \left(a_n\right)_{n \in \mathbb{N}_0} \text{hat HP} 1 \Rightarrow \limsup\\
		a_{2n+1} &= \frac{1}{2^{2n+1}} + \left(-1\right)^{2n+1} = \frac{1}{2^{2n} \cdot 2} - 1 = \frac{1}{4^n \cdot 2} -1 \xrightarrow{n\rightarrow\infty} -1 \\
		&\Rightarrow \left(a_n\right)_{n \in \mathbb{N}_0} \text{hat HP} -1 \Rightarrow \liminf
	\end{align*}
	
	\section{Cauchyfolgen}
	\begin{defi}
		Eine Folge $\left(a_n\right)_{n \in \mathbb{N}_0}$ hei"st Cauchyfolge genau dann, wenn
		\begin{center}
			$\forall \epsilon > 0$ $\exists n_0 \in \mathbb{N}$ $\forall n,m \geq n_0 : \left|a_n - a_m \right|$
		\end{center}
	\end{defi}
	\begin{bsp}
		Ist die Folge $a_n$ Cauchyfolge?
		\begin{center}
			$a_n = \frac{1}{n^2+n}$
		\end{center}
	\end{bsp}
	Es seien $n,m \in \mathbb{N}$ mit $n, m \geq n_0$ und $\epsilon > 0$ beliebig. O.B.d.A $n \geq m$\\
	\begin{align*}
		\left|a_m - a_n\right| &= \left|\frac{1}{m^2+m} - \frac{1}{n^2+n}\right| = \left|\frac{n^2+n-m^2-m}{\left(n^2+n\right) \left(m^2+m\right)}\right| \leq \left|\frac{n^2+n}{\left(n^2+n\right) \left(m^2+m\right)}\right|\\
		&\left|\frac{1}{m^2+m}\right| = \frac{1}{m^2+m} \leq \frac{1}{m} < \epsilon \Leftrightarrow \frac{1}{\epsilon} < m\\
		&\Rightarrow \text{Setze } n_0 = \left\lceil\frac{1}{\epsilon}\right\rceil + 1
	\end{align*}
	\begin{bsp}
		Ist die Folge $a_n$ Cauchyfolge?
		\begin{center}
			$a_n = \frac{7n^3-13n^2}{n^3} = 7 - \frac{13}{n}$
		\end{center}
	\end{bsp}
	Es seien $n,m \in \mathbb{N}$ mit $n, m \geq n_0$ und $\epsilon > 0$ beliebig. O.B.d.A $n \geq m$\\
	\begin{align*}
		\left|a_n - a_m\right| &= \left|7-\frac{13}{n}-\left(7-\frac{13}{m}\right)\right| = \left|\frac{13}{m} - \frac{13}{n}\right| \leq \left|\frac{13}{m} + \frac{13}{n}\right|\\
		&\overset{\triangle\text{-Ungl.}}{\leq} \left|\frac{13}{m}\right| + \left|\frac{13}{n}\right| \leq \frac{13}{m} + \frac{13}{m} = \frac{26}{m} < \epsilon \Leftrightarrow \frac{26}{\epsilon} < m\\
		&\Rightarrow \text{Setze } n_0 = \left\lceil\frac{26}{\epsilon}\right\rceil + 1
	\end{align*}
	
	\section{Reihen}
	\begin{defi}
		Sei $\left(a_n\right)_{n \in \mathbb{N}}$ eine Folge. Setze $s_n = \sum_{k = m}^{n} a_k, k \in \mathbb{N}$.\\
		Dann hei"st $s_n$ die $n$-te Partialsumme von $\left(a_n\right)_{n \in \mathbb{N}}$, die Folge der Partialsummen $\left(s_n\right)_{n \in \mathbb{N}}$ unendliche Reihe, bezeichnet $\sum_{n = m}^{\infty} a_n$.
	\end{defi}
	
	\subsection{Wichtige Reihen:}
	\begin{itemize}
		\item Harmonische Reihe: $\sum_{k = 1}^{\infty} \frac{1}{k}$ divergiert
		\item Geometrische Reihe: $\sum_{k = 0}^{\infty} q^k = \frac{1}{1-q}$ konvergiert bei $\left|q\right| < 1$
		\item $\sum_{k=1}^{\infty} \frac{1}{k^\alpha} = \begin{cases}
			< \infty, \alpha > 1 \Rightarrow \text{konvergiert} \\ 
			= \infty, \alpha \leq 1 \Rightarrow \text{divergiert}
		\end{cases}$
	\end{itemize}
	\newpage

	\subsection{Konvergenzkriterien}
	\subsubsection{Notwendiges Kriterium / Trivialkriterium}
	\begin{defi}
		Sei $\sum_{n=m}^{\infty} a_n$ konvergent, dann gilt $a_n \xrightarrow{n \rightarrow \infty} 0$
	\end{defi}
	\underline{Achtung:} nur notwendig, nicht hinreichend
	\begin{center}
		$\Rightarrow$ eignet sich nur zum Zeigen von Divergenz (dann, wenn Folge nicht gegen 0 konvergiert)
	\end{center}
	\begin{bsp}
		\begin{center}
			$\sum_{n = 1}^{\infty} \frac{n}{n+1}$, $a_n = \frac{n}{n+1} \xrightarrow{n \rightarrow \infty} 1$, also nicht 0 \\
			$\Rightarrow$ Reihe divergiert
		\end{center}
	\end{bsp}

	\subsubsection{Leibnizkriterium}
	\begin{defi}
		Sei $\left(a_n\right)_{n \in \mathbb{N}}$ eine reele, monoton fallende Nullfolge mit $a_n > 0$, $\forall n \in \mathbb{N}$.\\
		Dann konvergiert die alternierende Reihe $\sum_{n = m}^{\infty} \left(-1\right)^n a_n$
	\end{defi}
	\begin{bsp}
		\begin{center}
			$\sum_{k=1}^{\infty} \left(-1\right) ^k \frac{1}{k}$\\
			$\Rightarrow a_k = \frac{1}{k} > \frac{1}{k+1} = a_{k+1}, \frac{1}{k} > 0 \forall k \in \mathbb{N} \frac{1}{k} \xrightarrow{k \rightarrow \infty} 0$\\
			$\Rightarrow$ Reihe konvergiert
		\end{center}
	\end{bsp}
	\underline{Aber:} keine absolute Konvergenz, da $\sum_{k=1}^{\infty} \left|\left(-1\right)^k \frac{1}{k}\right| = \sum_{k=1}^{\infty} \frac{1}{k} = \infty$, da harmonische Reihe
	\newpage
	
	\subsubsection{Wurzelkriterium}
	\begin{defi}
		Sei $\sum_{n = m}^{\infty} a_n$ eine Reihe
		\begin{enumerate}[i)]
			\item Falls $\limsup \limits_{n \rightarrow \infty} \sqrt[n]{\left|a_n\right|} < 1$, dann konvergiert die Reihe absolut
			\item Falls $\limsup \limits_{n \rightarrow \infty} \sqrt[n]{\left|a_n\right|} > 1$, dann divergiert die Reihe
			\item Falls $\limsup \limits_{n \rightarrow \infty} \sqrt[n]{\left|a_n\right|} = 1$, dann lässt sich keine Aussage treffen
		\end{enumerate}
	\end{defi}
	\begin{bsp}
		\begin{center}
			$\sum_{n=1}^{\infty} \frac{n}{5^n}$\\
			$\Rightarrow \limsup \limits_{n \rightarrow \infty} \sqrt[n]{\left|a_n\right|} = \limsup \limits_{n \rightarrow \infty} \sqrt[n]{\frac{n}{5^n}} = \limsup \limits_{n \rightarrow \infty} \frac{\sqrt[n]{n}}{5} = \frac{1}{5} < 1 $\\
			$\Rightarrow$ Reihe konvergiert absolut
		\end{center}
	\end{bsp}

	\subsubsection{Quotientenkriterium}
	\begin{defi}
		Sei $\sum_{n = m}^{\infty} a_n$ eine Reihe, $a_n \neq 0$, $\forall n \geq m$
		\begin{enumerate}[i)]
			\item  Falls $\limsup \limits_{n \rightarrow \infty} \left| \frac{a_{n+1}}{a_n}\right| < 1$, dann konvergiert die Reihe absolut
			\item  Falls $\limsup \limits_{n \rightarrow \infty} \left| \frac{a_{n+1}}{a_n}\right| \geq 1$, dann divergiert die Reihe
		\end{enumerate}
	\end{defi}
	\newpage
	\begin{bsp}
		\begin{center}
			$\sum_{n=1}^{\infty} \frac{n^3}{2^n}$
		\end{center}
		\begin{align*}
			&\Rightarrow \limsup \limits_{n \rightarrow \infty} \left| \frac{a_{n+1}}{a_n}\right| = \limsup \limits_{n \rightarrow \infty} \left| \frac{\frac{\left( n+1\right)^3}{2^{n+1}}}{\frac{n^3}{2^n}}\right|\\
			&= \limsup \limits_{n \rightarrow \infty} \left| \frac{\left( n+1\right)^3}{2^{n+1}} \cdot \frac{2^n}{n^3}\right|\\
			&= \Rightarrow \limsup \limits_{n \rightarrow \infty} \left| \frac{a_{n+1}}{a_n}\right| = \limsup \limits_{n \rightarrow \infty} \left| \frac{\frac{\left( n+1\right)^3}{2^{n+1}}}{\frac{n^3}{2^n}}\right|\\ 
			&= \limsup \limits_{n \rightarrow \infty} \frac{\left(n+1\right)^3}{2n^3} = \limsup \limits_{n \rightarrow \infty} \frac{n^3 + 3n^2 + 3n + 1}{2n^3} = \limsup \limits_{n \rightarrow \infty} \frac{1 + \frac{3}{n} + \frac{3}{n^2} + \frac{1}{n^3}}{2}\\
			&= \frac{1}{2} < 1\\
			&\Rightarrow \text{Reihe konvergiert absolut}			
		\end{align*}
	\end{bsp}

	\subsubsection{Minorantenkriterium}
	\begin{defi}
		Seien $\left(a_n\right)_{n \in \mathbb{N}}$, $\left(b_n\right)_{n \in \mathbb{N}}$ Folgen mit $0 \leq a_n \leq b_n$. Ist $\sum_{n=m}^{\infty} a_n$ divergent, so divergiert auch $\sum_{n=m}^{\infty} b_n$.\\
		$a_n$ ist dann die Minorante.
	\end{defi}
	\begin{bsp}
		\begin{center}
			$\underbrace{\sum_{n=2}^{\infty} \frac{2n+1}{n^2-1}}_{:= b_n}$ 
		\end{center}
		\begin{center}
			$\frac{2n+1}{n^2-1} \geq \frac{2n}{n^2} = \frac{2}{n} \geq 0$, $\forall n \geq 2$\\
			$\Rightarrow 2 \cdot \sum_{n=2}^{\infty} \frac{1}{n}$ ist die harmonische Reihe und divergiert\\
			$\Rightarrow$ Reihe $b_n$ divergiert
		\end{center}
	\end{bsp}
	\newpage
	
	\subsubsection{Majorantenkriterium}
	\begin{defi}
		Seien $\left(a_n\right)_{n \in \mathbb{N}}$, $\left(b_n\right)_{n \in \mathbb{N}}$ Folgen mit $\left|a_n\right| \leq b_n$. Ist $\sum_{n=m}^{\infty} b_n$ konvergent, so konvergiert auch $\sum_{n=m}^{\infty} a_n$ und es gilt $\sum_{n=m}^{\infty} \left|a_n\right| < \sum_{n=m}^{\infty} b_n$.\\
		$b_n$ ist dann die Majorante.
	\end{defi}
	\begin{bsp}
		\begin{center}
			$\sum_{n=1}^{\infty} \frac{\sqrt{n}}{n^2+1}$
		\end{center}
		\begin{align*}
			\Rightarrow &a_n = \frac{\sqrt{n}}{n^2+1}\\
			&\left| \frac{\sqrt{n}}{n^2+1} \right| =  \frac{\sqrt{n}}{n^2+1} \leq \frac{\sqrt{n}}{n^2} = \frac{n^{\frac{1}{2}}}{n^2} = n^{\frac{1}{2}-2} = n^{-\frac{3}{2}} = \frac{1}{n^{\frac{3}{2}}}\\
			\Rightarrow &\text{ Mit } \alpha = \frac{3}{2} > 1 \text{ gilt nach Definition } \sum_{n=1}^{\infty} \frac{1}{n^{\frac{3}{2}}} \text{ konvergiert.}\\
			\Rightarrow &\text{Reihe konvergiert absolut.}
		\end{align*}
	\end{bsp}

	\subsection{Aufgaben}
		\begin{align*}
			&\text{i) } \sum_{n=1}^{\infty} \frac{1}{\frac{1}{2} \left|\sin\left(n\right)\right| + \frac{1}{2}n} &\text{iii) } \sum_{k=0}^{\infty} \frac{\left(-1\right)^k}{\sqrt[k]{k}}\\
			&\text{ii) } \sum_{k=0}^{\infty} \left(\frac{n}{n+1}\right)^{n^2} &\text{iv) } \sum_{k=1}^{\infty} \frac{\left(-1\right)^k}{k!}
		\end{align*}
\end{document}