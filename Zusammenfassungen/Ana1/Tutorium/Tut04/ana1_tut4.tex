\documentclass[a4paper,12pt]{article}
\usepackage{fancyhdr}
\usepackage[ngerman,german]{babel}
\usepackage{german}
\usepackage[utf8]{inputenc}
\usepackage[active]{srcltx}
\usepackage{algorithm}
\usepackage[noend]{algorithmic}
\usepackage{amsmath}
\usepackage{amssymb}
\usepackage{amsthm}
\usepackage{bbm}
\usepackage{enumerate}
\usepackage{graphicx}
\usepackage{ifthen}
\usepackage{listings}
\usepackage{struktex}
\usepackage{hyperref}
\usepackage{tabularx}
\usepackage[framemethod=tikz]{mdframed}
\usepackage{extarrows}
\usepackage{xcolor}
\definecolor{darkblue}{RGB}{0,0,102}
\definecolor{darkred}{RGB}{153,0,0}
\definecolor{darkgreen}{RGB}{0,102,0}

\newcommand{\Fach}{Analysis I}
\newcommand{\Name}{Lukas Harsch}
\newcommand{\Matrikelnummer}{979272}
\newcommand{\Semester}{Sommersemester 2019}
\newcommand{\Kapitel}{4}
\newcommand{\Titel}{}

% Task
\mdtheorem[
linecolor=darkblue,
frametitlefont=\sffamily\bfseries\color{white},
frametitlebackgroundcolor=darkblue,
]{bsp}{Beispiel}[section]

%Tipp
\mdtheorem[
linecolor=blue,
frametitlefont=\sffamily\bfseries\color{white},
frametitlebackgroundcolor=darkred,
]{tipp}{Hinweis}[section]

%Definition
\mdtheorem[
linecolor=darkgreen,
frametitlefont=\sffamily\bfseries\color{white},
frametitlebackgroundcolor=darkgreen,
]{defi}{Definition}[section]

\setlength{\parindent}{0em}
\topmargin 0cm
\oddsidemargin 0cm
\evensidemargin 0cm
\setlength{\textheight}{9.2in}
\setlength{\textwidth}{6.0in}

\hypersetup{
	pdftitle={\Fach{}: Tutorium \Kapitel{}},
	pdfauthor={\Name},
	pdfborder={0 0 0}
}

\lstset{ %
	language=java,
	basicstyle=\footnotesize\tt,
	showtabs=false,
	tabsize=2,
	captionpos=b,
	breaklines=true,
	extendedchars=true,
	showstringspaces=false,
	flexiblecolumns=true,
}

\title{Tutorium \Kapitel}
\author{\Name{}}

\begin{document}
	\thispagestyle{fancy}
	\lhead{\sf \small \Name{}}
	\chead{\sf \small \Fach}
	\rhead{\sf \small \Semester{}}
	\rfoot{\sf \tiny Keine Gewähr auf Richtigkeit und Vollständigkeit}
	\lfoot{\sf \tiny CC BY-NC-SA}
	\begin{center}
		\LARGE \sf \textbf{Tutorium \Kapitel}\\
		\LARGE \sf \Titel
	\end{center}
	\vspace*{0.2cm}
	
	\begin{enumerate}[i)]
		\item Die Funktion $f: \mathbb{Z} \rightarrow \mathbb{Z}$, $n \mapsto 2n$ ist injektiv, surjektiv oder bijektiv?
		\begin{itemize}
			\item injektiv: $f\left(n_1\right) = f\left(n_2\right) \Rightarrow 2n_1 = 2n_2 \Leftrightarrow n_1 = n_2$
			\item nicht surjektiv: für $y = 3 \in \mathbb{Z}$ $\nexists n$, sodass $f\left(n\right) = 3$
		\end{itemize}
		\item $\left(a_n\right)_{n \in \mathbb{N}}$ konvergiert $\Leftrightarrow$ Teilfolgen $\left(a_{2n}\right)_{n \in \mathbb{N}}$, $\left(a_{2n+1}\right)_{n \in \mathbb{N}}$ konvergiert. Wahr oder \underline{Falsch}?
		\begin{itemize}
			\item $\Rightarrow a_n = \left(-1\right)^n \Rightarrow a_{2n} = 1$, $a_{2n+1} -1$   
		\end{itemize}
		\item $\sum_{n=0}^{\infty} a_n$ konvergiert $\Leftrightarrow \left(a_n\right)_{n \in \mathbb{N}}$ eine Nullfolge ist. Wahr oder \underline{Falsch}?
		\begin{itemize}
			\item z.B. Harmonische Reihe
		\end{itemize}
	\end{enumerate}
	
	\section{Cauchyprodukt}
	\begin{defi}
		Seien $\sum_{k=0}^{\infty} a_k$ und $\sum_{k=0}^{\infty} b_k$ zwei absolut konvergierende Reihen für $k \in \mathbb{N}_0$.\\
		Sei $c_k := \sum_{j=0}^{k} a_k \cdot b_{k-j}$.\\
		
		$\Rightarrow$ Dann konvergiert auch $\sum_{k=0}^{\infty} c_k$ absolut und es gilt\\
		$\sum_{k=0}^{\infty} c_k = \left(\sum_{k=0}^{\infty} a_k\right) \cdot \left(\sum_{k=0}^{\infty} b_k\right)$\\
		
		\underline{Achtung:} Die Umkehrung gilt im Allgemeinen nicht!
	\end{defi}
	\begin{bsp}
		\begin{align*}
			&e^x = \sum_{k=0}^{\infty} \frac{x^k}{k!} &e^y = \sum_{k=0}^{\infty} \frac{y^k}{k!}
		\end{align*}
		\begin{center}
			Zu zeigen: $e^x \cdot e^y = e^{x+y}$
		\end{center}
	\end{bsp}
	\begin{align*}
		e^x \cdot e^y &= \sum_{k=0}^{\infty} \frac{x^k}{k!} \cdot \sum_{k=0}^{\infty} \frac{y^k}{k!}\\
		&\overset{\text{\tiny CP}}{=} \sum_{n=0}^{\infty} \underbrace{\sum_{k=0}^{\infty} \frac{x^k}{k!} \cdot \frac{y^{n-k}}{\left(n-k\right)!}}_{= c_k}\\
		&= \sum_{n=0}^{\infty} \sum_{k=0}^{n} \frac{1}{k! \cdot \left(n-k\right)!} \cdot x^k \cdot y^{n-k} \cdot \underbrace{\frac{n!}{n!}}_{\text{Nahrhafte 1}}\\
		&= \sum_{n=0}^{\infty} \underbrace{\sum_{k=0}^{n} \binom{n}{k} \cdot x^k \cdot y^{n-k}}_{\text{Binom. Lehrsatz }:= \left(x-y\right)^n} \cdot \frac{1}{n!}\\
		&\overset{\text{\tiny Bin}}{\underset{\text{\tiny LS}}{=}} \sum_{n=0}^{\infty} \left(x+y\right)^n \cdot \frac{1}{n!}\\
		&= \sum_{n=0}^{\infty} \frac{\left(x+y\right)^n}{n!}\\
		&\overset{\text{\tiny Def}}{\underset{\text{\tiny Exp}}{=}} e^{x+y}
	\end{align*}
	
	\section{Potenzreihe}
\end{document}