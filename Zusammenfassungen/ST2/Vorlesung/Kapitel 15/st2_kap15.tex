\documentclass[a4paper,12pt]{article}
\usepackage{fancyhdr}
\usepackage[ngerman,german]{babel}
\usepackage{german}
\usepackage[utf8]{inputenc}
\usepackage[active]{srcltx}
\usepackage{algorithm}
\usepackage[noend]{algorithmic}
\usepackage{amsmath}
\usepackage{amssymb}
\usepackage{amsthm}
\usepackage{bbm}
\usepackage{enumerate}
\usepackage{graphicx}
\usepackage{ifthen}
\usepackage{listings}
\usepackage{struktex}
\usepackage{hyperref}

\newcommand{\Fach}{Softwaretechnik II}
\newcommand{\Name}{Lukas Harsch}
\newcommand{\Matrikelnummer}{979272}
\newcommand{\Semester}{Sommersemester 2019}
\newcommand{\Kapitel}{15}
\newcommand{\Titel}{Aufwandsschätzung}

\setlength{\parindent}{0em}
\topmargin 0cm
\oddsidemargin 0cm
\evensidemargin 0cm
\setlength{\textheight}{9.2in}
\setlength{\textwidth}{6.0in}

\hypersetup{
	pdftitle={\Fach{}: Kapitel \Kapitel{} - \Titel},
	pdfauthor={\Name},
	pdfborder={0 0 0}
}

\lstset{ %
	language=java,
	basicstyle=\footnotesize\tt,
	showtabs=false,
	tabsize=2,
	captionpos=b,
	breaklines=true,
	extendedchars=true,
	showstringspaces=false,
	flexiblecolumns=true,
}

\title{Kapitel \Kapitel}
\author{\Name{}}

\begin{document}
	\thispagestyle{fancy}
	\lhead{\sf \small \Name{}}
	\chead{\sf \small \Fach}
	\rhead{\sf \small \Semester{}}
	\rfoot{\sf \tiny Keine Gewähr auf Richtigkeit und Vollständigkeit}
	\lfoot{\sf \tiny Grafiken aus den Vorlesungsfolien\\CC BY-NC-SA }
	\begin{center}
		\LARGE \sf \textbf{Kapitel \Kapitel}\\
		\LARGE \sf \Titel
	\end{center}
	\vspace*{0.2cm}
	
	\section{Projekt}
	\subsection*{Softwarequalität}
	\begin{itemize}
		\item quality in use - wie gut ist die Anwendung langfristig wenn in Benutzung?
		\item external quality - Testfälle, Qualität der Anwendung
		\item internal quality - Codequalität
		\item process quality - wie arbeite ich?
	\end{itemize}
	\begin{center}
		$\Rightarrow$ untere Punkte beeinflussen Obere
	\end{center}
	\section{Projektmanagement}
	\subsection*{Erfolgsfaktoren für ein Projekt}
	\begin{itemize}
		\item Planung, Arbeitsteilung
		\item gute/eingehaltene Prozesse
		\item Controlling
		\item Kontakt zum Auftraggeber
		\item Team
		\item Verantwortlichkeit
	\end{itemize}

	\section{Aufwandsschätzung}
	\subsection*{Ermittlung des späteren Verkaufspreis}
	Um Kosten im Projekt (also Personalkosten, Hardware- und Softwarekosten, Reise- und Ausbildungskosten) zu tilgen und dazu noch Gewinn erzielen zu können, muss der spätere Verkaufspreis richtig kalkuliert werden. Wie kann man einen solchen Verkaufspreis jetzt ermitteln, am Beispiel einer "`Where the fuck is my Bus"' App für die SWU?
	\begin{itemize}
		\item Analogie: Gibt es bereits ähnliche Produkte, an denen îch mich festhalten kann?
		\item Daten: Anzahl der Nutzer, Anzahl der Busse
		\item Fertigstellungsfrist: Bis wann muss die Anwendung fertig entwickelt sein?
		\item Funktionalität: Was muss die Anwendung alles können?
		\item Dokumentation: Lastenheft und Pflichtenheft
		\item Budget des Kunden
	\end{itemize}
\end{document}